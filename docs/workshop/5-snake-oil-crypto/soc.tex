\documentclass{beamer}
\usetheme[numbering=progressbar]{focus}
\usepackage{tikz}
\usepackage{listings}
\usetikzlibrary{positioning}
\usetikzlibrary{shapes,arrows}
\usepackage{transparent}
\usepackage{fancyvrb}
\usepackage{listings}
\definecolor{main}{RGB}{47, 161, 219}
%\definecolor{textcolor}{RGB}{128, 128, 128}
\definecolor{background}{RGB}{240, 247, 255}
\definecolor{textcolor}{RGB}{85, 87, 83}
\title{Snake Oil Crypto:}
\subtitle{How I stopped to worry and started to love crypto}
\author{Jean-Louis Huynen}
\titlegraphic{\includegraphics[scale=0.20]{../../logos/d4-logo.pdf}}
\institute{Team CIRCL \\ }
\date{2019/12/06}

\begin{document}

\begin{frame}
    \maketitle
\end{frame}

\begin{frame}
        \frametitle{Outline}

        \begin{itemize}
          \item Use-Case: RSA,
          \item First Hands-on: Understanding RSA,
          \item Snake-Oil-Crypto: a primer,
          \item Second Hands-on: RSA in Snake-Oil-Crypto,
          \item D4 passiveSSL Collection,
          \item Interactions with MISP.
        \end{itemize}

\end{frame}

\input{rsabasics.tex}

\input{soc-d4.tex}

\input{d4-pssl.tex}

\begin{frame}
  \frametitle{First release}
  \begin{itemize}
  \item[\checkmark] sensor-d4-tls-fingerprinting
    \footnote{\url{github.com/D4-project/sensor-d4-tls-fingerprinting}}:
    {\bf Extracts} and {\bf fingerprints} certificates, and {\bf computes} TLSH fuzzy hash.
  \item[\checkmark] analyzer-d4-passivessl
    \footnote{\url{github.com/D4-project/analyzer-d4-passivessl}}:
    {\bf Stores} Certificates / PK details in a PostgreSQL DB.
  \item snake-oil-crypto 
    \footnote{\url{github.com/D4-project/snake-oil-crypto}}:
    {\bf Performs} crypto checks, push results in MISP for notification
  \item lookup-d4-passivessl
    \footnote{\url{github.com/D4-project/lookup-d4-passivessl}}:
    {\bf Exposes} the DB through a public REST API.
  \end{itemize} 
\end{frame}

\begin{frame}
\frametitle{Get in touch if you want to join/support the project, host a passive ssl sensor or contribute}
\begin{itemize}
\item Collaboration can include research partnership, sharing of collected streams or improving the software.
\item Contact: info@circl.lu
\item \url{https://github.com/D4-Project} -  \url{https://twitter.com/d4_project}
\end{itemize}
\end{frame}

\end{document}
