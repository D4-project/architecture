% Full instructions available at:
% https://github.com/elauksap/focus-beamertheme

\documentclass{beamer}
\usetheme[numbering=progressbar]{focus}
\definecolor{main}{RGB}{47, 161, 219}
%\definecolor{textcolor}{RGB}{128, 128, 128}
\definecolor{background}{RGB}{240, 247, 255}
\definecolor{textcolor}{RGB}{85, 87, 83}
\title{D4 Project}
\subtitle{Open and collaborative network monitoring}
\author{Alexandre Dulaunoy - Sami Mokaddem}
\titlegraphic{\includegraphics[scale=0.20]{d4-logo.pdf}}
\institute{Team CIRCL \\ \url{https://www.d4-project.org/}}
\date{20190207}

\begin{document}
    \begin{frame}
        \maketitle
    \end{frame}
%    \section{Section 1}


\begin{frame}
        \frametitle{Problem statement}
        \begin{itemize}
                \item CSIRTs (or private organisations) build their {\bf own honeypot, honeynet or blackhole monitoring network}.
                \item Designing, managing and operating such infrastructure is a tedious and resource intensive task.
                \item {\bf Automatic sharing} between monitoring networks from different organisations is missing.
                \item Sensors and processing are often seen as blackbox or difficult to audit.

        \end{itemize}
\end{frame}


\begin{frame}
 \frametitle{Objective}
 \begin{itemize}
         \item Based on our experience with MISP\footnote{\url{https://github.com/MISP/MISP}} where sharing played an important role, we transpose
                 the model in D4 project.
         \item Keeping the protocol and code base {\bf simple and minimal}.
         \item Allowing every organisation to {\bf control and audit their own sensor network}.
         \item Extending D4 or {\bf encapsulating legacy monitoring protocols} must be as simple as possible.
         \item Ensuring that the sensor server has {\bf no control on the sensor} (unidirectional streaming).
 \end{itemize}
\end{frame}


\end{document}
